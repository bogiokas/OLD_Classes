\documentclass[12pt]{article}
\usepackage{amsmath,amsthm,amssymb}
\usepackage{geometry}
\usepackage{makeidx}
\usepackage{graphicx}
\usepackage{color}
\usepackage[usenames,dvipsnames]{xcolor}
\usepackage{tikz}
\usepackage{xfrac}
\usepackage{array}
\usepackage{color}
\usepackage{listings}
\lstset{ %
language=C++,
basicstyle=\footnotesize,
numbers=left,
numberstyle=\footnotesize,
stepnumber=1,
numbersep=5pt,
backgroundcolor=\color{white},
showspaces=false,% show spaces adding particular underscores
showstringspaces=false,% underline spaces within strings
showtabs=false,% show tabs within strings adding particular underscores
tabsize=2,% sets default tabsize to 2 spaces
captionpos=b,% sets the caption-position to bottom
breaklines=true,% sets automatic line breaking
breakatwhitespace=false,
escapeinside={\%*}{*)}          % if you want to add a comment within your code
}
\usepackage{xstring}

\usepackage[english,greek]{babel}
\usepackage[utf8x]{inputenc}
%\usepackage{ucs}

\usepackage{enumerate}
\usepackage{enumitem}
\setlist{nolistsep}
\usepackage{hyperref}
\hypersetup{colorlinks=true,linkcolor=MidnightBlue}

\newcommand\en[1]{\latintext #1\greektext}
\newcommand\m[1]{\mbox{$\displaystyle #1 $}}
\newcommand\ul[1]{\emph{#1}}
\newcommand\bigOh{\mathcal{O}}
\newcommand\nospace{\hspace*{-0.5em}}

\renewcommand{\thefootnote}{*}

\newenvironment{n_enum}{
\begin{enumerate}[label=(\arabic{*})]
  \setlength{\itemsep}{0pt}
  \setlength{\parskip}{0pt}
  \setlength{\parsep}{0pt}
}{\end{enumerate}}

\newenvironment{i_enum}{
\begin{enumerate}[label=(\roman{*})]
  \setlength{\itemsep}{0pt}
  \setlength{\parskip}{0pt}
  \setlength{\parsep}{0pt}
}{\end{enumerate}}

\newenvironment{a_enum}{
\begin{enumerate}[label=(\alph{*})]
  \setlength{\itemsep}{0pt}
  \setlength{\parskip}{0pt}
  \setlength{\parsep}{0pt}
}{\end{enumerate}}

\newenvironment{b_item}{
\begin{itemize}
  \setlength{\itemsep}{0pt}
  \setlength{\parskip}{0pt}
  \setlength{\parsep}{0pt}
}{\end{itemize}}

\newcommand{\HRule}{\rule{\linewidth}{0.1mm}}

\begin{document}
\begin{center}
{\bf Αλγόριθμοι και πολυπλοκότητα}\\
4η Σειρά Γραπτών ασκήσεων
\end{center}
Χειμερινό Εξάμηνο 2013-2014 \hfill Μπογιόκας Δημήτριος - ΜΠΛΑ
\HRule\\
{\bf Άσκηση 1}\\\HRule\\
{\bf Άσκηση 2}\\\HRule\\
{\bf Άσκηση 3}\\\HRule\\
{\bf Άσκηση 4}(α) Αρχικά ισχύει ότι $3Partition\in NP$. Πράγματι, σε μια δεδομένη διαμέριση χρειάζεται γραμμικός χρόνος για να σχηματισθούν τα επί μέρους αθροίσματα και να ελεγχθεί το κατά πόσον ισούνται μεταξύ τους. Στη συνέχεια θα δείξω ότι η 3-διαμέριση είναι $NP$-\en{hard}, δείχνοντας ότι
$$Partition\leq_m^p3Partition$$
Έστω $\mathcal{A}=\{w_1,w_2,\ldots,w_n\}$, όπου $n\in\mathbb{N}$ $w_1,\ldots,w_n\in\mathbb{Z}_{>0}$ και $2|\sum_{i=1}^nw_i$ μία είσοδος του προβλήματος $Partition$. Ορίζω τη συνάρτηση $f$ ως εξής:
$$f(\mathcal{A}):=\left\{w_1,w_2,\ldots,w_n,\frac{\sum_{i=1}^nw_i}{2}\right\}$$
για όλες τις πιθανές εισόδους $\mathcal{A}$. Τότε, το $f(\mathcal{A})$ είναι είσοδος του $3Partition$, χρειάζεται πολυωνυμικό χρόνο για να υπολογισθεί και επιπλέον ισχύει
$$\mathcal{A}\in Partition\Leftrightarrow f(\mathcal{A})\in3Partition$$
Πράγματι, έστω $\{w_1,\ldots,w_n\}\in Partition$. Δηλαδή, υπάρχουν ξένα υποσύνολα δεικτών $I,J\subseteq\{1,\ldots,n\}$ με $I\cup J=\{1,\ldots,n\}$ τέτοια ώστε:
$$\sum_{i\in I}w_i=\sum_{j\in J}w_j=b$$
όπου $b=\frac{1}{2}\sum_{i=1}^nw_i$. Άρα τα σύνολα $\{w_i:i\in I\},\{w_j:j\in J\},\{b\}$ αποτελούν προφανώς τριμέριση του $f(\{w_1,\ldots,w_n\})$ με τις επιθυμητές ιδιότητες. Άρα $f(\{w_1,\ldots,w_n\})\in3Partition$. Αντίστροφα έστω κάποιο $\mathcal{A}=\{w_1,\ldots,w_n\}$ τέτοιο ώστε $f(\mathcal{A})\in3Partition$. Τότε, αν $b$ όπως πριν, κάθε ένα από τα 3 σύνολα της διαμέρισης έχει άθροισμα $b$ και αφού το $b$ είναι στοιχείο του αρχικού συνόλου, ένα από τα τρία σύνολα πρέπει να είναι το μονοσύνολο $\{b\}$. Άρα υπάρχουν υποσύνολα δεικτών $I,J$ όπως πρίν, ώστε
$$\sum_{i\in I}w_i=\sum_{j\in J}w_j=b$$
άρα, προφανώς $\{w_1,\ldots,w_n\}\in Partition$.\\\HRule\\
{\bf Άσκηση 4}(β) Αρχικά ισχύει ότι το πρόβλημα (θα το συμβολίζω με $STLeafs$) ανήκει στο $NP$. Πράγματι, ένα δεδομένο υποσύνολο ακμών μπορεί να ελεγχθεί σε πολυωνυμικό χρόνο ως προς το αν σχηματίζει \en{spanning tree} και κατά πόσον τα φύλλα αυτού είναι υποσύνολο δεδομένου συνόλου κορυφών $L$. Στη συνέχεια θα δείξω ότι το $STLeafs$ είναι $NP$-\en{hard}, δείχνοντας ότι
$$HamPath\leq_m^pSTLeafs$$
όπου $HamPath$ είναι το πρόβλημα ύπαρξης μονοπατιού \en{Hamilton}. Έστω $G=(V,E)$ μια είσοδος του προβλήματος $HamPath$. Ορίζω τη συνάρτηση $f$ ως εξής:
$$f(V,E)=\{G'=(V',E'),L=\{u_1,u_2\}\}$$
όπου
$$V'=V\cup\{v_1,v_2,u_1,u_2\}$$
και
$$E'=E\cup\{\{v_1,v\}:v\in E\}\cup\{\{v_2,v\}:v\in E\}\cup\{\{v_1,u_1\},\{v_2,u_2\}\}$$
θεωρώντας ότι οι κορυφές $u_1,u_2,v_1,v_2$ που προσθέτω δεν ανήκουν στο αρχικό $V$. Ουσιαστικά η συνάρτηση παίρνει ως είσοδο ένα γράφημα, του προσθέτει αρχικά δύο κορυφές $v_1,v_2$ που τις ενώνει με όλες τις κορυφές του αρχικού γραφήματος και στη συνέχεια προσθέτει δύο κορυφές βαθμού $1$, τις $u_1,u_2$ που τις ενώνει μόνο με την $v_1$ και τη $v_2$ αντιστοίχως. Τέλος, η είσοδος του $STLeafs$ αποτελείται από το $G'$ και από το $L=\{u_1,u_2\}\subseteq V'$. Η $f$ προφανώς είναι πολυωνυμικά υπολογίσιμη και επιπλέον ισχύει
$$G\in HamPath\Leftrightarrow f(G)\in STLeafs$$
Πράγματι, έστω $G$ γράφημα, τέτοιο ώστε $G\in HamPath$. Δηλαδή, υπάρχουν κορυφές $s,t\in V$ και σύνολο ακμών $Κ=\{\{s=r_1,r_2\},\{r_2,r_3\},\ldots,\{r_{n-1},r_n=t\}\}$ με $r_i\neq r_j$ όταν $i\neq j$ και επιπλέον $\cup_{i=1}^n\{r_i\}=V$. Σχηματίζω το σύνολο $K'\subseteq E'$ ως εξής:
$$K'=K\cup\{\{s,v_1\},\{t,v_2\},\{v_1,u_1\},\{v_2,u_2\}\}$$
Το $G'[K']$ (το υπογράφημα του $G'$ που επάγεται από το $K'$ είναι προφανώς \en{spanning tree} του $G'=f(G)$ και κάθε κορυφή στο $G'[K']$, εκ κατασκευής, έχει βαθμό $2$, εκτός από τις $u_1,u_2$ που είναι άρα τα μοναδικά φύλλα του δέντρου. Δηλαδή ισχύει $\{u_1,u_2\}\subseteq L$ (για την ακρίβεια τα σύνολα ταυτίζονται). Άρα $f(G)\in STLeafs$. Αντίστροφα, έστω $G$ γράφημα, τέτοιο ώστε $G'=f(G)\in STLeafs$, δηλαδή υπάρχει \en{spanning tree} $T$ του $G'$, τέτοιο ώστε τα φύλλα του να είναι υποσύνολο του $\{u_1,u_2\}$. Αφού τα $u_1,u_2$ έχουν βαθμό $1$ στο $G'$, θα έχουν βαθμό $\leq1$ και σε κάθε υπογράφημα, ειδικότερα στο εν λόγω \en{spanning tree}. Δηλαδή τα φύλλα του $T$ είναι ακριβώς οι κορυφές $u_1,u_2$. Είναι αρκετά εύκολο να δείξει κάποιος επαγωγικά ότι σε ένα δέντρο ισχύει ότι το πλήθος των φύλλων είναι τουλάχιστον τόσο όσο ο βαθμός της μέγιστης κορυφής. Αφού το $T$ έχει δύο φύλλα, κάθε κορυφή του $T$ έχει βαθμό το πολύ $2$. Άρα, κάθε κορυφή του $T$, εκτός από τις $u_1,u_2$ έχει βαθμό ακριβώς $2$. Δηλαδή, το $T$ είναι \en{Hamilton Path} στο $G'$. Άρα, αν $s,t$ είναι οι γείτονες των $v_1$ και $v_2$, αντιστοίχως, μέσα στο $G$, τότε το $T\setminus\{\{v_1,u_1\},\{v_2,u_2\},\{v_1,s\},\{v_2,t\}\}$ είναι \en{Hamilton Path} στο $G$, δηλαδή $G\in HamPath$.\\\HRule\\
{\bf Άσκηση 4}(γ) Αρχικά ισχύει ότι το πρόβλημα (έστω $FVSD$) ανήκει στο $NP$. Πράγματι, δεδομένου ενός υποσυνόλου $S$ του συνόλου των κορυφών ενός γραφήματος $G=(V,E)$, σε πολυωνυμικό χρόνο μπορεί να διαπιστωθεί κατά πόσον το $G[V\setminus S]$ είναι ακυκλικό, μέσω του αλγορίθμου \en{DFS}. Στη συνέχεια θα δείξω ότι το $FVSD$ είναι $NP$-\en{hard}, δείχνοντας ότι
$$VC\leq_m^p FVSD$$
Έστω $G$ ένα μη κατευθυνόμενο γράφημα και $(G,k)$ μία είσοδος του προβλήματος $VC$. Ορίζω τη συνάρτηση $f$ ως εξής:
$$f((V,E),k)=(G',k)$$
όπου $G'=(V',E')$ ένα κατευθυνόμενο γράφημα, τέτοιο ώστε:
$$V'=V\qquad\qquad E'=\{(u,v):\{u,v\}\in E\}$$
Παρατηρώ ότι με τον παραπάνω συμβολισμό που χρησιμοποιώ, εννοείται ότι αν $\{u,v\}\in E$, τότε τόσο η $(u,v)$, όσο και η $(v,u)$ ανήκουν στο $E'$. Ουσιαστικά, το γράφημα $G'$ παράγεται από το $G$ αντικαθιστώντας κάθε ακμή $\{u,v\}$ του $G$ με τις (κατευθυνόμενες) ακμές $(u,v)$ και $(v,u)$. Η $f$ προφανώς είναι υπολογίσιμη σε γραμμικό χρόνο και επιπλέον ισχύει:
$$(G,k)\in VC\Leftrightarrow f(G,k)\in FVSD$$
Πράγματι, έστω γράφημα $G=(V,E)$ και $k\in\mathbb{Z}_{>0}$ τέτοια ώστε $(G,k)\in VC$, υπάρχει δηλαδή $Κ\subseteq V$, υπούνολο των κορυφών του $G$, τέτοιο ώστε $|K|\leq k$ και για κάθε ακμή $e$ του $G$ να ισχύει $e\cap K\neq\emptyset$. Τότε, προφανώς το $K$ έχει την επιθυμητή ιδιότητα στο $G'$, αφού το γράφημα $G[V'\setminus K]$ δεν περιέχει ακμές, πόσο μάλλον κύκλους. Άρα, ισχύει $f(G,k)\in FVSD$. Αντίστροφα, έστω γράφημα $G$ και $k\in\mathbb{Z}_{>0}$ τέτοια ώστε $f(G,k)\in FVSD$, τότε υπάρχει ένα $K\subseteq V$ τέτοιο ώστε να ακουμπάει κάθε κατευθυνόμενο κύκλο στο $G'$ και $|Κ|\leq k$. Όμως για κάθε $\{u,v\}\in E(G)$, στο $G'$ υπάρχει ο κατευθυνόμενος κύκλος $\{(u,v),(v,u)\}$, άρα τουλάχιστον ένα από τα $v,u$ ανήκουν στο $K$, δηλαδή το $K$ είναι κάλυμμα κορυφών του γραφήματος $G$, άρα $(G,k)\in VC$.\\\HRule\\
{\bf Άσκηση 4}(δ) Ισχύει ότι το πρόβλημα (στο εξής θα λέγεται $CSP$) ανήκει στο $NP$. Πράγματι, ένα σύνολο ακμών ελέγχεται πολυωνυμικά για το αν είναι $s-t$ μονοπάτι και για το κατά πόσον το συνολικό κόστος διέλευσης και ο συνολικός χρόνος διέλευσης είναι μικρότερα ή ίσα από $M$ και $T$ αντιστοίχως. Στη συνέχεια θα δείξω ότι το $CSP$ είναι και $NP$-\en{hard}, αποδεικνύοντας ότι
$$Knapsack\leq_m^p CSP$$
Έστω $((w_1,v_1),(w_2,v_2),\ldots,(w_n,v_n),W,V)$ μια είσοδος του προβλήματος $Knapsack$. Δηλαδή το $(w_i,v_i)$ είναι ζεύγος βάρους-αξίας του εκάστοτε αντικειμένου, $W$ είναι το μέγιστο δυνατό βάρος και $V$ είναι η ελάχιστη δυνατή αξία. Ορίζω τη συνάρτηση $f$ ως εξής:
$$f((w_1,v_1),(w_2,v_2),\ldots,(w_n,v_n),W,V)=(G,a,b,m,t,Μ,Τ)$$
όπου $G=(V,E)$ κατευθυνόμενο γράφημα, $a,b\in V$ συγκεκριμένες κορυφές του $G$ και $m,t:E\to\mathbb{N}$ συναρτήσεις ορισμένες στις ακμές και $M,T\in\mathbb{N}$, τα οποία κατασκευάζονται ως εξής:
(Χρησιμοποιώ τον συμβολισμό $[n]=\{1,2,\ldots,n\}$)
$$V=\{a,b\}\cup\{x_{ij}:i,j\in[n],i\geq j\}$$
$$E=\{(a,b)\}\cup\{(a,x_{i1}):i\in[n]\}\cup\{(x_{ij},b):i,j\in[n],i\geq j\}\cup\{(x_{ij},x_{kl}):i<k,j=l-1\}$$
$$m((s,t))=\left\{\begin{array}{rcl}
0&,&s=a\land t=b\\
w_i&,&s=a\land t=x_{i1}\\
0&,&s=x_{ij}\land t=b\\
w_k&,&s=x_{ij}\land t=x_{kl}\land i<k\land j=l-1\\
\end{array}\right.$$
Έστω $v=\max\{v_i:i\in[n]\}$
$$t((s,t))=\left\{\begin{array}{rcl}
nv&,&s=a\land t=b\\
iv-v_i&,&s=a\land t=v_{i1}\\
(n-i)v&,&s=x_{ij}\land t=b\\
(k-i)v-v_k&,&s=x_{ij}\land t=x_{kl}\land i<k\land j=l-1\\
\end{array}\right.$$
$$M=W\qquad\qquad\qquad T=nv-V$$
Ουσιαστικά, οι κόμβοι $x_{ij}$ του γραφήματος κωδικοποιούν το <<τοποθετώ το αντικείμενο $i$ στο σάκο, και είναι το $j$-οστό αντικείμενο που τοποθετώ στο σάκο>> (θεωρώ ότι τα αντικείμενα διατάσσονται σύμφωνα με το δείκτη που έχουν αρχικά και τοποθετούνται τελικά, μετά την επιλογή, με αυτή τη σειρά στο σάκο) οι κόμβοι $a$ και $b$ κωδικοποιούν την αρχή και το τέλος της διαδικασίας. Η συνάρτηση $m$ εκφράζει ακριβώς το βάρος κάθε αντικειμένου που βάζω (αναφέρεται στο αντικείμενο, στο οποίο φτάνει η ακμή), ενώ η συνάρτηση $t$ εκφράζει την ποσότητα $v-v_i$, δηλαδή θεωρώ ότι συνολικά υπάρχει αξία $nv$ εκτός του σάκου και με το αντικείμενο $i$ που τοποθετώ στο σάκο, αφαιρώ από αυτή την αξία ακριβώς $v_i$ (πάλι, όταν ανατίθεται σε μια ακμή αναφέρεται στην αξία του αντικειμένου στο οποίο φτάνει η ακμή). Όπου την ποσότητα $v$ την όρισα προηγουμένως ως το μέγιστο από τα δυνατά βάρη. (στην πραγματικότητα το $v$ θα μπορούσε εξ' ίσου καλά να είναι οποιοσδήποτε αριθμός μεγαλύτερος ή ίσος από κάθε βάρος) Τελικά, οι ακμές κωδικοποιούν τις δυνατές κινήσεις που μπορώ να κάνω:
\begin{b_item}
\item Να είμαι στον κόμβο $a$ (ο σάκος να είναι άδειος) και να πάω στον $b$ (να επιλέξω να μη βάλω κανένα αντικείμενο). Τότε το βάρος που προσθέτω είναι $0$ και η αξία που αφήνω, αφήνοντας ταυτόχρονα όλα τα αντικείμενα εκτός του σάκου είναι η μέγιστη δυνατή, δηλ. $nv$.
\item Να είμαι στον κόμβο $a$ και να πάω στον $x_{i1}$ (να επιλέξω να βάλω πρώτο το αντικείμενο $i$). Τότε, το βάρος αυξάνεται κατά $w_i$ και η επιπλεόν αξία που αφήνω είναι $v$ για κάθε ένα από τα $i-1$ αντικείμενα που δεν έβαλα και ακόμη $v-v_i$ για το αντικείμενο $i$, δηλαδή, συνολικά $iv-v_i$.
\item Να είμαι στον κόμβο $x_{ij}$ (έχω μόλις τοποθετήσει το αντικείμενο $i$, με την $j$-οστή τοποθέτηση) και να πάω στον κόμβο $b$. Τότε το βαρος που προσθέτω είναι $0$ και η αξία που αφήνω, αφήνοντας τα αντικείμενα $i+1,\ldots,n$ εκτός του σάκου είναι $(n-i)v$.
\item Να είμαι στον κόμβο $x_{ij}$ και να πάω στον κόμβο $x_{kl}$, όπου $k>i$, αφού διατρέχω τα αντικείμενα με συγκεκριμένη σειρά και $l=j+1$ αφού αναφέρομαι στην επόμενη τοποθέτηση. Τότε το βάρος που προσθέτω είναι $w_k$ και η αξία που αφήνω είναι $v$ για κάθε ένα από τα αντικείμενα $i+1,i+2,\ldots,k-2,k-1$, τα οποία είναι $k-i-1$ το πλήθος και ακόμη $v-v_k$ για το αντικείμενο $k$, δηλαδή συνολικά $(k-i)v-v_k$.
\end{b_item}
Τελικά εξετάζω αν υπάρχει μονοπάτι από το $a$ στο $b$ (από τον άδειo σάκο στον κλειστό σάκο), ώστε το συνολικό βάρος να μην ξεπερνάει το $M=W$ και η συνολική αξία που αφήνω εκτός του σάκου να μην ξεπερνάει το $nv-V$ (δηλαδή η αξία που θα πάρω μαζί να είναι τουλάχιστον $V$).
Η παραπάνω συνάρτηση είναι προφανώς πολυωνυμικά υπολογίσιμη και επίσης γι αυτήν ισχύει
$$((w_1,v_1),\ldots,(w_n,v_n),W,V)\in Knapsack\Leftrightarrow f((w_1,v_1),\ldots,(w_n,v_n),W,V)\in CSP$$
Πράγματι, έστω μια είσοδος του $Knapsack$ με τους παραπάνω συμβολισμούς, που να ανήκει στο $Knapsack$. Τότε υπάρχει επιλογή δεικτών $I\subseteq[n]$ για την οποία ισχύει $\sum_{i\in I}w_i\leq W$ και $\sum_{i\in I}v_i\geq V$. Έστω $I=\{i_1<i_2<\cdots<i_p\}$, τότε για το μονοπάτι $P=\{(a,x_{i_11}),(x_{i_11},x_{i_22}),\ldots,(x_{i_pp},b)\}$ ισχύει ότι
$$\sum_{e\in P}m(e)=\sum_{i\in I}w_i\leq W$$
από υπόθεση και ότι
$$\begin{array}{rcl}\sum_{e\in P}t(e)
&=&i_1v-v_{i_1}+\sum_{f=1}^{p-1}(i_{f+1}-i_f)v-v_{i_{f+1}}+(n-i_p)v\\
&=&i_1v-v_{i_1}+v\sum_{f=1}^{p-1}(i_{f+1}-i_f)-\sum_{f=1}^{p-1}v_{i_{f+1}}+nv-i_pv\\
&=&nv-\sum_{f=1}^pv_{i_f}\leq nv-V\\
\end{array}$$
από υπόθεση, επίσης. Άρα τελικά ισχύει
$$f((w_1,v_1),\ldots,(w_n,v_n),W,V)\in CSP$$
Αντίστροφα, υποθέτω μια είσοδο του $Knapsack$, τέτοια ώστε η εικόνα της μέσω της $f$ να ανήκει στο $CSP$. Τότε, η επιλογή των αντικειμένων που θα βάλω στο σάκο προκύπτει άμεσα από τα $i$ στους κόμβους $x_{ij}$ που χρησιμοποιούνται στο μονοπάτι. Η επαλήθευση ότι οι ανισότητες ισχύουν είναι το ίδιο τετριμμένη, όπως στην αντίστροφη κατεύθυνση που εξέτασα πριν. Άρα, τελικά ισχύει ότι η αρχική είσοδος ανήκει στο $Knapsack$, το οποίο ολοκληρώνει την απόδειξη. Ακραίες περιπτώσεις όπως να μη βάλει κάποιος τίποτα στον σάκο, επίσης καλύπτονται με ακρίβεια από την παραπάνω κατασκευή.


\end{document}
